\documentclass{article}
\usepackage[natbib]{biblatex}
\usepackage{indentfirst}



\addbibresource{bibliography.bib}

\title{Bibliography Summary}
\author{Minh Nguyen}
\date{March 22, 2022}

\begin{document}


\maketitle

\section*{Summary}
When it comes to mobile app coding, there are multiple IDEs to use, among the popular ones are Visual Studio Code and Android Studio. Visual Studio Code is so much lighter compared to Android Studio. Both IDE have the ability to search in while in the project. "Visual Studio Code is an editor that favors simplicity over having an endless assortment of bells and whistles" \cite{Lewis22}. Meanwhile, "Android Studio is more of a “kitchen sink” approach to an IDE" \cite{Lewis22}, the users can just pick and drag any tab in between or put onto a alternative screen which much more flexible than Visual Studio Code. If the users prefer fast and lightweight IDE for mobile app coding, they should try out Visual Studio Code. When it comes to using Git for your project, Android Studio has more flexibility than Visual Studio Code, ", Android Studio has quite a bit of flexibility when it comes to using Git as your source control system" \cite{Lewis22}. "In Android Studio, when creating a Flutter project you get the choice to create an application, package, plugin, or module. You can also choose the package name and platform channel language (Java/Kotlin and Obj-C/Swift). No such choices in VS Code" \cite{Suragch20}. There are different variety of functionality between the 2 IDEs, if you want light IDE, go for VS Code. Also, there are some plugins and enhancements only limited to one or the other, so it will impact the decision of which one to choose as well. \cite{Suragch20}




\printbibliography

\end{document}
