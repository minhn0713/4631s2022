\documentclass{article}
\usepackage[natbib]{biblatex}
\usepackage{indentfirst}



\addbibresource{bibliography.bib}

\title{Bibliography Summary}
\author{Minh Nguyen}
\date{April 27, 2022}

\begin{document}


\maketitle

\section*{Summary}
There are iOS, Android, and Web developers. But one platform development is more popular compared to the other platforms development. There are multiple differences among those three such as coding language if they aren't using native, specific design for each platform, popularity between iOS and Android \cite{Ego21}. The popularity of iOS is increasing in demand and it is really popular across the globe compared to Android. So the iOS developers is also increasing much more than Android developers \cite{Ego21}. But nowadays, we have React Native, Flutter, etc for developers to code cross-platform. It means like 1 code works for three platforms (iOS, Android, and Web). Developers have been migrating into React Native, Flutter, and similar platforms. Mobile apps is continue to grow and have more demand than ever.

\medskip
We are evolving everyday, so the world. There are 5 trends that will shape the future of mobile apps such as app discovery on steroids, On-Demand Stays in Big Demand, Organic Gains on Free-Time Gains, The Install-Inclined Will Continue to Decline, Freedom From Friction \cite{Jon21}. App discovery has been exploding since 2021, since people are at home during Covid and they have 4G/5G speed, they keep exploring app on their phones. "Research shows new device owners are keen to install new apps, a phenomenon we refer to as an 'app-alanche.'" \cite{Jon21}. For On-Demand apps such as groceries apps, shopping apps, social media apps, have been exploding as well since the pandemic started back in 2019. Organic apps and paid apps were on a huge spiked in downloads and usage during Covid. "Organic and paid installations were up 21 percent and 15 percent respectively, proving that when people have more free time, their go-to way to fill it is the apps on their device." \cite{Jon21}. Simply it's not because of Covid that started it, it has been already started before Covid, Covid is just the reason that helps accelerated it. For the people who install the new apps would decline, but the number of new apps installations will incline. In 2017, "the research company reported that 51 percent smartphone owners said they didn’t install apps", after 2 years later, they did the research again and the result went up to like 67 percent \cite{Jon21}. Since the Covid started, smartphones consumers perspective of smartphones has changed drastically. So the apps marketers can do something to make their consumers feel freedom when using their smartphones and mobile apps, "Smart app marketers can build on the consumer’s new view of their phones by incorporating things like playable ads and instant ads in their UA campaigns" \cite{Jon21}.





\printbibliography

\end{document}
