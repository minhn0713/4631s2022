\documentclass{article}
\usepackage[natbib]{biblatex}
\usepackage{indentfirst}



\addbibresource{bibliography.bib}

\title{Bibliography Summary}
\author{Minh Nguyen}
\date{February 9, 2022}

\begin{document}


\maketitle

\section*{Summary}
The mobile apps has blown up so much that every company have tracking embedded in their apps for showing ads based on users searches. Apple has implemented a new way to let their users base to opt out tracking from apps that are installed in Apple devices. This implementation from Apple tremendously shifting to another level of ad spending. Since when Apple implemented this, "The amount advertisers spend on iOS dropped around a third" \cite{oconnor2021}. This means that the advertisers has been decreasing rapidly and continuing to decrease if Apple do not make any changes regrading this issue. Meanwhile, the Android advertisers spend on Android increasing about 10 percents. "Recent data from PubMatic shows that the share of mobile app ad spend on Android has grown to 63 percent from 54 percent since Apple rolled out its Apple Ad Tracking" \cite{oconnor2021}. This basically shows that and advertisers from iOS are migrating to Android.

\medskip

Even though Apple implemented the 'Do Not Track' method for their users base, there are still multiple apps on iOS can get through this. "According to an investigation by researchers at privacy software maker Lockdown and The Washington Post. Subway Surfers starts sending an outside ad company called Chartboost 29 very specific data points about your iPhone, including your Internet address, your free storage, your current volume level (to 3 decimal points) and even your battery level (to 15 decimal points)" \cite{ran2021}. Subway Surfers is an gaming app, this investigation shows that this gaming app gone through Apple 'Do Not Track' method and they obtain specific data points on our Apple devices without our consents, they can obtain our information even though we opted for Do Not Track. "Among the apps Lockdown investigated, tapping the don’t track button made no difference at all to the total number of third-party trackers the apps reached out to. And the number of times the apps attempted to send out data to these companies declined just 13 percent." \cite{ran2021}. This seems ridiculous that the Apple's Do Not Track method made no difference to the third-party apps. There is nothing we can do to prevent this because either we opted for Do not track or we opted for do track, they will track you either options you chose unless Apple stepped in and take action against it and make some update to their so called Do not track method.

\printbibliography

\end{document}
