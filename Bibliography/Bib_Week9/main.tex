\documentclass{article}
\usepackage[natbib]{biblatex}
\usepackage{indentfirst}



\addbibresource{bibliography.bib}

\title{Bibliography Summary}
\author{Minh Nguyen}
\date{March 29, 2022}

\begin{document}


\maketitle

\section*{Summary}
Since the COVID-19 started in 2020, mobile apps industry has grown in a rapidly speed. One of the mobile apps industry that has absolutely blown up is fitness app. Fitness app has grown 67 percent in installs during the pandemic \cite{Robert20}. "In the U.S., installs of health and fitness apps were 58 percent higher at the end of March from the 2020 average. Daily sessions peaked at 25 percent above average in May, and have slowly declined since then. By July, daily sessions were 16 percent higher than average" \cite{Robert20}. It is actually good for their health since we got lockdown during the pandemic and people been working from home, so they tried any possible ways to get exercise at home so they decided to install health apps. Another app in the mobile apps industry has blown up since the pandemic started are social media apps. Since people were working at home, and currently some people working at home during the pandemic, not only health apps has blown up, but also every social media apps has blown up. "But beginning in mid-March, when statewide stay-at-home orders went into effect, social media app usage began to increase significantly, and now consumes around 25 percent of all mobile app usage for U.S. adults" \cite{Sara20}. When they working at home, they communicate with each other through social media apps, and also chat with their friends and long distance family on social media apps since during the pandemic they were on lockdown and have no way to see each other face to face. When using the social media apps, users can video chat and texting with each other so that is the reason why social media apps has blown up rapidly \cite{Sara20}.




\printbibliography

\end{document}
